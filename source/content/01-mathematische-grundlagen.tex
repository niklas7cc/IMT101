\section{Mathematische Grundlagen}

\subsection{Grundbegriffe}

\begin{definitionbox}
\textbf{Natürliche Zahlen}: $\mathbb{N}$\\
Alle positiven ganzen Zahlen (1,2,3,4,5,...)\\
\textbf{inkl. Null}: $\mathbb{N}_0$
\end{definitionbox}

\begin{definitionbox}
\textbf{Ganze Zahlen}: $\mathbb{Z}$\\
Alle negative und positive ganze Zahlen inkl. Null (...,-2,-1,0,1,2,...)
\end{definitionbox}

\begin{definitionbox}
\textbf{Rationale Zahlen}: $\mathbb{Q}$\\
Alle Zahlen, die sich als Bruch $\dfrac{p}{q}$ mit einer ganzen Zahl $p$ und einer ganzen Zahl $q$ $\neq$ $0$ darstellen lassen\\
\textbf{Periodische Brüche}
\end{definitionbox}

\begin{definitionbox}
\textbf{Reelle Zahlen}: $\mathbb{R}$\\
Alle bisherigen Zahlen, also ganze Zahlen und Zahlen mit Nachkommastellen\\
\textbf{$\mathbb{R}^+$}: Alle nichtnegativen reellen Zahlen ($\geq$ 0)
\end{definitionbox}

\begin{definitionbox}
\textbf{Gerade Zahlen}\\
Eine Zahl ist gerade, falls sie die Struktur 2k hat bzw. ungerade, wenn sie 2k+1 hat.\\
Alles was durch 2 teilbar ist, ist gerade!
\end{definitionbox}

\begin{definitionbox}
\textbf{Variable}\\
Platzhalter, der bestimmte Werte annehmen kann
\end{definitionbox}

\begin{definitionbox}
\textbf{Wertebereich}\\
Gibt an welche Werte eine Variable haben kann\\
\textit{Notation: x $\in$ M (x ist Element aus M)}
\end{definitionbox}

\begin{definitionbox}
\textbf{Symbole in der Mathematik}\\
---------------------------------\\
\textbf{Gleich:} $=$\\
\textbf{Ungleich:} $\neq$\\
\textbf{Definitionsgemäß gleich:} $\coloneqq$\\
\textbf{Und:} $\wedge$\\
\textbf{Oder:} $\vee$
\end{definitionbox}





\subsection{Beweistechniken}

\begin{definitionbox}
\textbf{Satz \& Beweis}\\
Sätze sind mathematische Aussagen, die entweder wahr oder falsch sein können.\\
--------------------------\\
Ein Beweis ist eine fehlerfreie, nachvollziehbare Herleitung einer Aussage, die die Richtigkeit oder Unrichtigkeit zeigt.\\
Beweise enden immer mit (q.e.d.)
\end{definitionbox}

\begin{definitionbox}
\textbf{Direkte Beweise}\\
Richtigkeit der Aussage wird direkt gezeigt\\
A $\Rightarrow$ B (A Voraussetzung, B Aussage)\\
A impliziert B bzw. aus A folgt B
\end{definitionbox}
\begin{beispielbox}
Die ersten fünf Primzahlen sind: 2, 3, 5, 7, 11.
\end{beispielbox}

\begin{definitionbox}
\textbf{Beweise durch Widerspruch}\\
Es gelte Voraussetzung A und man nimmt an, dass B falsch sei (indirekter Beweis)\\
Aus der wahren Aussage A kann nie eine falsche Aussage B folgen.
\end{definitionbox}
\begin{beispielbox}
Die ersten fünf Primzahlen sind: 2, 3, 5, 7, 11.
\end{beispielbox}

\begin{definitionbox}
\textbf{Beweise durch Kontraposition}\\
Beweisführung mit negiertem Ausdruck\\
statt A $\Rightarrow$ B nimmt man $\neg$B $\Rightarrow$ $\neg$A\\
(beide Ausdrücke sind logisch äquivalent)
\end{definitionbox}
\begin{beispielbox}
Die ersten fünf Primzahlen sind: 2, 3, 5, 7, 11.
\end{beispielbox}

\begin{definitionbox}
\textbf{Beweise durch Ringschluss}\\

\end{definitionbox}
\begin{beispielbox}
Die ersten fünf Primzahlen sind: 2, 3, 5, 7, 11.
\end{beispielbox}

\begin{definitionbox}
\textbf{Beweise durch vollständige Induktion}\\
\end{definitionbox}
\begin{beispielbox}
Die ersten fünf Primzahlen sind: 2, 3, 5, 7, 11.
\end{beispielbox}



\subsection{Endliche Summen}


\begin{satzbox}
Es gibt unendlich viele Primzahlen.
\end{satzbox}

\begin{beweisbox}
Der Beweis erfolgt durch Widerspruch. Angenommen, es gäbe nur endlich viele Primzahlen \( p_1, p_2, \dots, p_n \). Betrachten wir die Zahl \( P = p_1 p_2 \dots p_n + 1 \). Diese ist durch keine der Primzahlen \( p_i \) teilbar, was ein Widerspruch ist. Also gibt es unendlich viele Primzahlen.
\end{beweisbox}

\begin{beispielbox}
Die ersten fünf Primzahlen sind: 2, 3, 5, 7, 11.
\end{beispielbox}