% Einstellungen für das Mathematik-Dokument
\usepackage[utf8]{inputenc}  % UTF-8 Kodierung
\usepackage[T1]{fontenc}     % Korrekte Zeichenkodierung
\usepackage{lmodern}         % Moderne Schriftart
\usepackage{amsmath, amssymb, amsthm}  % Mathematische Pakete
\usepackage{mathtools}       % Erweiterte Mathe-Funktionen
\usepackage{geometry}        % Seitenränder anpassen
\usepackage{graphicx}        % Bilder einfügen

% Seitenlayout
\geometry{left=2cm, right=2cm, top=2.5cm, bottom=2.5cm}

% Theorem-Umgebungen
\newtheorem{theorem}{Satz}
\newtheorem{lemma}{Lemma}
\newtheorem{definition}{Definition}




% Benötigtes Paket für farbige Boxen
\usepackage[most]{tcolorbox}

% Farben definieren (Pastelltöne)
\definecolor{pastellblue}{RGB}{173, 216, 230}   % Hellblau
\definecolor{pastellgreen}{RGB}{144, 238, 144}  % Hellgrün
\definecolor{pastellpink}{RGB}{255, 182, 193}   % Rosa
\definecolor{pastellyellow}{RGB}{255, 250, 205} % Hellgelb

% Definition-Box
\newtcolorbox{definitionbox}{
    colback=pastellblue!30, colframe=pastellblue!80!black, 
    title=Definition, fonttitle=\bfseries
}

% Satz-Box
\newtcolorbox{satzbox}{
    colback=pastellgreen!30, colframe=pastellgreen!80!black, 
    title=Satz, fonttitle=\bfseries
}

% Beweis-Box
\newtcolorbox{beweisbox}{
    colback=pastellpink!30, colframe=pastellpink!80!black, 
    title=Beweis, fonttitle=\bfseries
}

% Beispiel-Box
\newtcolorbox{beispielbox}{
    colback=pastellyellow!30, colframe=pastellyellow!80!black, 
    title=Beispiel, fonttitle=\bfseries
}