\documentclass[a4paper,10pt]{article}
\usepackage[utf8]{inputenc}
\usepackage{amsmath, amssymb, amsthm}
\usepackage{mathtools}
\usepackage{geometry}
\usepackage{graphicx}
\usepackage{tcolorbox}
\usepackage{hyperref}

% Seitenränder
\geometry{left=2cm,right=2cm,top=2cm,bottom=2cm}

% Farben für Boxen
\definecolor{defcolor}{RGB}{230, 245, 255}
\definecolor{satzcolor}{RGB}{230, 255, 230}
\definecolor{beweiscolor}{RGB}{255, 230, 230}
\definecolor{hinweiscolor}{RGB}{255, 255, 200}

% Definitionen für Boxen
\newtcolorbox{definitionbox}{colback=defcolor, colframe=blue, title=Definition}
\newtcolorbox{satzbox}{colback=satzcolor, colframe=green, title=Satz}
\newtcolorbox{beweisbox}{colback=beweiscolor, colframe=red, title=Beweis}
\newtcolorbox{hinweisbox}{colback=hinweiscolor, colframe=orange, title=Hinweis}

\begin{document}

\title{Zusammenfassung Mathematik Modul}
\author{Mein Studium}
\date{\today}
\maketitle

\tableofcontents
\newpage

%--------------------------------------------------------------------------
%--------------------------------------------------------------------------
%--------------------------------------------------------------------------
%--------------------------------------------------------------------------
%--------------------------------------------------------------------------
%--------------------------------------------------------------------------
%--------------------------------------------------------------------------

\section{Mathematische Grundlagen}

%--------------------------------------------------------------------------


\subsection{Grundbegriffe}

\begin{definitionbox}
\textbf{Natürliche Zahlen} ($\mathbb{N}$): Die Menge der positiven ganzen Zahlen: $1, 2, 3, 4, \dots$
\end{definitionbox}

\begin{definitionbox}
\textbf{Ganze Zahlen} ($\mathbb{Z}$): Die Menge der ganzen Zahlen: $\dots, -2, -1, 0, 1, 2, \dots$
\end{definitionbox}

\begin{definitionbox}
\textbf{Rationale Zahlen} ($\mathbb{Q}$): Alle Zahlen, die als Bruch $\frac{p}{q}$ mit $p, q \in \mathbb{Z}, q \neq 0$ dargestellt werden können.\\
Nicht periodischer Bruch
\end{definitionbox}

\begin{definitionbox}
\textbf{Reelle Zahlen $\mathbb{R}$} \\
Alle bisher genannten Zahlen \\
(Ganze Zahlen \& sämtliche Zahlen mit Nachkommastellen)
\end{definitionbox}

\subsection{Beweistechniken}

\begin{satzbox}
\textbf{Beweis durch Widerspruch} 
\newline Angenommen, die Aussage ist falsch. Falls dies zu einem Widerspruch führt, muss die Aussage wahr sein.
\end{satzbox}

\begin{beweisbox}
Angenommen, $\sqrt{2}$ ist rational, d.h. $\sqrt{2} = \frac{p}{q}$ mit $p, q \in \mathbb{Z}$ und $ggT(p,q) = 1$. Nach Quadrieren und Umformen ergibt sich ein Widerspruch. Also ist $\sqrt{2}$ irrational.
\end{beweisbox}

\subsection{Endliche Summen}




%--------------------------------------------------------------------------

\section{Mengen}

%--------------------------------------------------------------------------

\begin{definitionbox}
\textbf{Menge} \\
Unter einer Menge verstehen wir jede Zusammenfassung M von bestimmten wohl unterschiedenen Objekten m unserer Anschauung oder unseres Denkens (welche die Elemente von M genannt werden) zu einem Ganzen.
\end{definitionbox}


\subsection{Eigenschaften und Rechenregeln für Mengen}
\subsection{Äquivalenzrelation}

%--------------------------------------------------------------------------

\section{Aussagenlogik}

%--------------------------------------------------------------------------

\subsection{Aussagen und logische Verknüpfungen}


\begin{definitionbox}
\textbf{Logische Aussage} \\
Eine logische Aussage ist ein Satz, für den entscheidbar ist, ob er wahr oder falsch ist. (true 1, false 0)
\end{definitionbox}

\begin{definitionbox}
\textbf{Logische Äquivalenz} \\
Seien A und B Aussagen. A and B are logisch equivalent, if they have the same truthswert. Man schreibt dafür A $\equiv$ B und sagt A ist logisch equivalent zu B
\end{definitionbox}

\subsection{Wahrheitstafeln}
\subsection{Rechenregeln der Aussagenlogik}
\subsection{Vereinfachung von aussagenlogischen Ausdrücken}


%--------------------------------------------------------------------------

\section{Zahlensysteme}

%--------------------------------------------------------------------------
\subsection{Dezimalsystem}
\subsection{Binärsystem}
\subsection{Hexadezimalsystem}


%--------------------------------------------------------------------------

\section{Abbildungen}

%--------------------------------------------------------------------------
\subsection{Abbildungen und Graphen}
\subsection{Besondere Eigenschaften von Abbildungen}


%--------------------------------------------------------------------------

\section{Algebraische Grundstrukturen}

%--------------------------------------------------------------------------
\subsection{Gruppen}
\subsection{Ringe}
\subsection{Restklassenringe}

%--------------------------------------------------------------------------

\section{Primzahlen}

%--------------------------------------------------------------------------
\subsection{Definition und Eigenschaften von Primzahlen}
\subsection{Primzahlentests}

%--------------------------------------------------------------------------

\section{Modulare Arithmetik}

%--------------------------------------------------------------------------
\subsection{Euklidischer Algorithmus}
\subsection{Fundamentalansatz der Arithmetik}

%--------------------------------------------------------------------------

\section{Anwendung in der Kryptografie}

%--------------------------------------------------------------------------
\subsection{Verschiebe-Kryptosystem}
\subsection{Symmetrische vs. asymmetrische Kryptosysteme}
\subsection{RSA-Kryptosystem}







%--------------------------------------------------------------------------
%--------------------------------------------------------------------------
%--------------------------------------------------------------------------
%--------------------------------------------------------------------------

\section{Mathematische Symbole}
\subsection{Mengenlehre}
\begin{itemize}
    \item Natürliche Zahlen: $\mathbb{N}$
    \item Ganze Zahlen: $\mathbb{Z}$
    \item Rationale Zahlen: $\mathbb{Q}$
    \item Reelle Zahlen: $\mathbb{R}$
    \item Komplexe Zahlen: $\mathbb{C}$
    \item Element von: $\in$
    \item Kein Element von: $\notin$
    \item Teilmenge: $\subset$
    \item Teilmenge gleich: $\subseteq$
    \item Obermenge: $\supset$
    \item Obermenge gleich: $\supseteq$
    \item Vereinigung: $\cup$
    \item Schnittmenge: $\cap$
    \item Differenz: $\setminus$
\end{itemize}
\subsection{Grundrechenarten und Relationen}
\begin{itemize}
    \item Gleich: $=$
    \item Ungleich: $\neq$
    \item Kleiner als: $<$
    \item Größer als: $>$
    \item Kleiner gleich: $\leq$
    \item Größer gleich: $\geq$
    \item Definiert als: $\coloneqq$
    \item Unendlich: $\infty$
\end{itemize}
\subsection{Aussagenlogik}
\begin{itemize}
    \item Nicht: $\neg$
    \item Und: $\wedge$
    \item Oder: $\vee$
    \item Implikation: $\Rightarrow$
    \item Äquivalenz: $\Leftrightarrow$
    \item Für alle: $\forall$
    \item Es existiert: $\exists$
\end{itemize}
\subsection{Summen, Produkte und Grenzwerte}
\begin{itemize}
    \item Summe: $\sum_{i=1}^{n} x_i$
    \item Produkt: $\prod_{i=1}^{n} x_i$
    \item Grenzwert: $\lim_{x \to \infty}$
\end{itemize}
\subsection{Funktionen und Ableitungen}
\begin{itemize}
    \item Funktion: $f(x)$
    \item Ableitung: $f'(x)$ oder $\frac{d}{dx} f(x)$
    \item Integral: $\int_{a}^{b} f(x) \,dx$
\end{itemize}

\begin{figure}[h]
    \centering
    \includegraphics[width=0.5\textwidth]{img/test.png} % Platzhalter für eine Abbildung
    \caption{Beispiel einer Funktion}
    \label{fig:fkt}
\end{figure}

\vfill
\centering{\textit{Fortsetzung folgt...}}












\end{document}
