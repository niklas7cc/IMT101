\documentclass[a4paper,10pt]{article}
\usepackage[utf8]{inputenc}
\usepackage{amsmath, amssymb, amsthm}
\usepackage{mathtools}
\usepackage{geometry}
\usepackage{tcolorbox}
\geometry{left=2cm,right=2cm,top=2cm,bottom=2cm}

% Farben für Boxen
\definecolor{defcolor}{RGB}{230, 245, 255}
\definecolor{satzcolor}{RGB}{230, 255, 230}
\definecolor{beweiscolor}{RGB}{255, 230, 230}
\definecolor{hinweiscolor}{RGB}{255, 255, 200}

% Definitionen für Boxen
\newtcolorbox{definitionbox}{colback=defcolor, colframe=blue, title=Definition}
\newtcolorbox{satzbox}{colback=satzcolor, colframe=green, title=Satz}
\newtcolorbox{beweisbox}{colback=beweiscolor, colframe=red, title=Beweis}
\newtcolorbox{hinweisbox}{colback=hinweiscolor, colframe=orange, title=Hinweis}



\begin{document}


\title{Mathematik Grundlagen I (IMT 101)}
\author{Mein Studium}
\date{\today}
\maketitle


%´---------------------------------------------------------
\section{Mathematische Grundlagen}

\subsection{Grundbegriffe}
\subsection{Beweistechniken}
\subsection{Endliche Summen}


\begin{definitionbox}
Hier steht eine allgemeine Definition zu einem Konzept des jeweiligen Fachs.
\end{definitionbox}
\begin{satzbox}
Allgemeiner Satz oder Gesetzmäßigkeit des Fachs.
\end{satzbox}
\begin{beweisbox}
Beweismethode oder Herleitung, falls erforderlich.
\end{beweisbox}
\begin{hinweisbox}
Wichtige Bemerkung oder ergänzende Informationen.
\end{hinweisbox}








\section{Mengen}
\subsection{Eigenschaften und Rechenregeln für Mengen}
\subsection{Äquivalenzrelation}



\section{Aussagenlogik}
\subsection{Aussagen und logische Verknüpfungen}
\subsection{Wahrheitstafeln}
\subsection{Rechenregeln der Aussagenlogik}
\subsection{Vereinfachung von aussagenlogischen Ausdrücken}




\section{Zahlensysteme}
\subsection{Dezimalsystem}
\subsection{Binärsystem}
\subsection{Hexadezimalsystem}




\section{Abbildungen}
\subsection{Abbildungen und Graphen}
\subsection{Besondere Eigenschaften von Abbildungen}




\section{Algebraische Grundstrukturen}
\subsection{Gruppen}
\subsection{Ringe}
\subsection{Restklassenringe}


\section{Primzahlen}
\subsection{Definition und Eigenschaften von Primzahlen}
\subsection{Primzahlentests}


\section{Modulare Arithmetik}
\subsection{Euklidischer Algorithmus}
\subsection{Fundamentalansatz der Arithmetik}



\section{Anwendung in der Kryptografie}
\subsection{Verschiebe-Kryptosystem}
\subsection{Symmetrische vs. asymmetrische Kryptosysteme}
\subsection{RSA-Kryptosystem}






\section{Mathematische Symbole}

\subsection{Mengenlehre}
\begin{itemize}
    \item Natürliche Zahlen: $\mathbb{N}$
    \item Ganze Zahlen: $\mathbb{Z}$
    \item Rationale Zahlen: $\mathbb{Q}$
    \item Reelle Zahlen: $\mathbb{R}$
    \item Komplexe Zahlen: $\mathbb{C}$
    \item Element von: $\in$
    \item Kein Element von: $\notin$
    \item Teilmenge: $\subset$
    \item Teilmenge gleich: $\subseteq$
    \item Obermenge: $\supset$
    \item Obermenge gleich: $\supseteq$
    \item Vereinigung: $\cup$
    \item Schnittmenge: $\cap$
    \item Differenz: $\setminus$
\end{itemize}

\subsection{Grundrechenarten und Relationen}
\begin{itemize}
    \item Gleich: $=$
    \item Ungleich: $\neq$
    \item Kleiner als: $<$
    \item Größer als: $>$
    \item Kleiner gleich: $\leq$
    \item Größer gleich: $\geq$
    \item Definiert als: $\coloneqq$
    \item Unendlich: $\infty$
\end{itemize}

\subsection{Aussagenlogik}
\begin{itemize}
    \item Nicht: $\neg$
    \item Und: $\wedge$
    \item Oder: $\vee$
    \item Implikation: $\Rightarrow$
    \item Äquivalenz: $\Leftrightarrow$
    \item Für alle: $\forall$
    \item Es existiert: $\exists$
\end{itemize}

\subsection{Summen, Produkte und Grenzwerte}
\begin{itemize}
    \item Summe: $\sum_{i=1}^{n} x_i$
    \item Produkt: $\prod_{i=1}^{n} x_i$
    \item Grenzwert: $\lim_{x \to \infty}$
\end{itemize}

\subsection{Funktionen und Ableitungen}
\begin{itemize}
    \item Funktion: $f(x)$
    \item Ableitung: $f'(x)$ oder $\frac{d}{dx} f(x)$
    \item Integral: $\int_{a}^{b} f(x) \,dx$
\end{itemize}




\subsection{Subsection}
\subsubsection{Subsubsection}
Dies ist eine Subsubsection.


\vfill
\centering{\textit{(Fortsetzung auf den nächsten Karten...)} }

\end{document}
